
\documentclass[10pt,a4paper]{report}
\usepackage[latin1]{inputenc}
\usepackage{amsmath}
\usepackage{amsfonts}
\usepackage{amssymb}
\usepackage{graphicx}
\usepackage{multicol}
\usepackage{tabularx}

\usepackage{hyperref}
\usepackage{tikz}
\usetikzlibrary{matrix,calc}
\usepackage[margin=0.5in]{geometry}
\newenvironment{Figure}
  {\par\medskip\noindent\minipage{\linewidth}}
  {\endminipage\par\medskip}
\begin{document}
\begin{figure*}[!tbp]
  \centering
  \begin{minipage}[b]{0.4\textwidth}
    \includegraphics[scale = 0.05]{iith_logo.jpg}
  \end{minipage}
  \hfill
\end{figure*}


\raggedright \textbf{Name}:\hspace{1mm} Lakshmi Kamakshi\hspace{3cm} \Large \textbf{ASSIGNMENT-1}\hspace{2.5cm} %
\normalsize \textbf{Roll No.} :\hspace{1mm} FWC22019\vspace{2cm}
\begin{multicols}{2}
\hspace{7.5cm} \textbf{Karnaugh Maps}\vspace{0.5cm}
\raggedright \\The process of simplifying the algebraic expression of a boolean function is called minimization. Minimization is important since it reduces the cost and complexity of the associated circuit.The minimization of the algebraic expressions can be done in two ways:\\1.Mimization using algebraic manipulation\\2.Minimization using K-map \vspace{3mm} \\ \begin{itemize}


\item We can minimize Boolean expressions of 3, 4 variables very easily using K-map without using any Boolean algebra theorems.
\item K-map can take two forms Sum of Product (SOP) and Product of Sum (POS) according to the need of problem. 
\item K-map is table like representation but it gives more information than TRUTH TABLE.
\item We fill grid of K-map with $0's $and $1's$ then solve it by making groups.\vspace{2mm}
\end{itemize}
\raggedright \textbf{Problem Statement:}\vspace{2mm}
\raggedright \\Reduce the following boolean expression to it's simplest form using K-Map:
\center F(X,Y,Z,W) = $\sum(0,1,2,3,4,5,10,11,14)$
\vspace{5mm}



%isolated term
%#1 - Optional. Space between node and grouping line. Default=0
%#2 - node
%#3 - filling color
\newcommand{\implicantsol}[3][0]{
    \draw[rounded corners=3pt, fill=#3, opacity=0.3] ($(#2.north west)+(135:#1)$) rectangle ($(#2.south east)+(-45:#1)$);
    }


%internal group
%#1 - Optional. Space between node and grouping line. Default=0
%#2 - top left node
%#3 - bottom right node
%#4 - filling color
\newcommand{\implicant}[4][0]{
    \draw[rounded corners=3pt, fill=#4, opacity=0.3] ($(#2.north west)+(135:#1)$) rectangle ($(#3.south east)+(-45:#1)$);
    }

%group lateral borders
%#1 - Optional. Space between node and grouping line. Default=0
%#2 - top left node
%#3 - bottom right node
%#4 - filling color
\newcommand{\implicantcostats}[4][0]{
    \draw[rounded corners=3pt, fill=#4, opacity=0.3] ($(rf.east |- #2.north)+(90:#1)$)-| ($(#2.east)+(0:#1)$) |- ($(rf.east |- #3.south)+(-90:#1)$);
    \draw[rounded corners=3pt, fill=#4, opacity=0.3] ($(cf.west |- #2.north)+(90:#1)$) -| ($(#3.west)+(180:#1)$) |- ($(cf.west |- #3.south)+(-90:#1)$);
}

%group top-bottom borders
%#1 - Optional. Space between node and grouping line. Default=0
%#2 - top left node
%#3 - bottom right node
%#4 - filling color
\newcommand{\implicantdaltbaix}[4][0]{
    \draw[rounded corners=3pt, fill=#4, opacity=0.3] ($(cf.south -| #2.west)+(180:#1)$) |- ($(#2.south)+(-90:#1)$) -| ($(cf.south -| #3.east)+(0:#1)$);
    \draw[rounded corners=3pt, fill=#4, opacity=0.3] ($(rf.north -| #2.west)+(180:#1)$) |- ($(#3.north)+(90:#1)$) -| ($(rf.north -| #3.east)+(0:#1)$);
}

%group corners
%#1 - Optional. Space between node and grouping line. Default=0
%#2 - filling color
\newcommand{\implicantcantons}[2][0]{
    \draw[rounded corners=3pt, opacity=.3] ($(rf.east |- 0.south)+(-90:#1)$) -| ($(0.east |- cf.south)+(0:#1)$);
    \draw[rounded corners=3pt, opacity=.3] ($(rf.east |- 8.north)+(90:#1)$) -| ($(8.east |- rf.north)+(0:#1)$);
    \draw[rounded corners=3pt, opacity=.3] ($(cf.west |- 2.south)+(-90:#1)$) -| ($(2.west |- cf.south)+(180:#1)$);
    \draw[rounded corners=3pt, opacity=.3] ($(cf.west |- 10.north)+(90:#1)$) -| ($(10.west |- rf.north)+(180:#1)$);
    \fill[rounded corners=3pt, fill=#2, opacity=.3] ($(rf.east |- 0.south)+(-90:#1)$) -|  ($(0.east |- cf.south)+(0:#1)$) [sharp corners] ($(rf.east |- 0.south)+(-90:#1)$) |-  ($(0.east |- cf.south)+(0:#1)$) ;
    \fill[rounded corners=3pt, fill=#2, opacity=.3] ($(rf.east |- 8.north)+(90:#1)$) -| ($(8.east |- rf.north)+(0:#1)$) [sharp corners] ($(rf.east |- 8.north)+(90:#1)$) |- ($(8.east |- rf.north)+(0:#1)$) ;
    \fill[rounded corners=3pt, fill=#2, opacity=.3] ($(cf.west |- 2.south)+(-90:#1)$) -| ($(2.west |- cf.south)+(180:#1)$) [sharp corners]($(cf.west |- 2.south)+(-90:#1)$) |- ($(2.west |- cf.south)+(180:#1)$) ;
    \fill[rounded corners=3pt, fill=#2, opacity=.3] ($(cf.west |- 10.north)+(90:#1)$) -| ($(10.west |- rf.north)+(180:#1)$) [sharp corners] ($(cf.west |- 10.north)+(90:#1)$) |- ($(10.west |- rf.north)+(180:#1)$) ;
}

%Empty Karnaugh map 4x4
\newenvironment{Karnaugh}%
{
\begin{tikzpicture}[baseline=(current bounding box.north),scale=0.8]
\draw (0,0) grid (4,4);
\draw (0,4) -- node [pos=0.7,above right,anchor=south west] {ZW} node [pos=0.7,below left,anchor=north east] {XY} ++(135:1);
%
\matrix (mapa) [matrix of nodes,
        column sep={0.8cm,between origins},
        row sep={0.8cm,between origins},
        every node/.style={minimum size=0.3mm},
        anchor=8.center,
        ampersand replacement=\&] at (0.5,0.5)
{
                       \& |(c00)| 00         \& |(c01)| 01         \& |(c11)| 11         \& |(c10)| 10         \& |(cf)| \phantom{00} \\
|(r00)| 00             \& |(0)|  \phantom{0} \& |(1)|  \phantom{0} \& |(3)|  \phantom{0} \& |(2)|  \phantom{0} \&                     \\
|(r01)| 01             \& |(4)|  \phantom{0} \& |(5)|  \phantom{0} \& |(7)|  \phantom{0} \& |(6)|  \phantom{0} \&                     \\
|(r11)| 11             \& |(12)| \phantom{0} \& |(13)| \phantom{0} \& |(15)| \phantom{0} \& |(14)| \phantom{0} \&                     \\
|(r10)| 10             \& |(8)|  \phantom{0} \& |(9)|  \phantom{0} \& |(11)| \phantom{0} \& |(10)| \phantom{0} \&                     \\
|(rf) | \phantom{00}   \&                    \&                    \&                    \&                    \&                     \\
};
}%
{
\end{tikzpicture}
}




%Defines 8 or 16 values (0,1,X)
\newcommand{\contingut}[1]{%
\foreach \x [count=\xi from 0]  in {#1}
     \path (\xi) node {\x};
}

%Places 1 in listed positions
\newcommand{\minterms}[1]{%
    \foreach \x in {#1}
        \path (\x) node {1};
}

%Places 0 in listed positions
\newcommand{\maxterms}[1]{%
    \foreach \x in {#1}
        \path (\x) node {0};
}

%Places X in listed positions
\newcommand{\indeterminats}[1]{%
    \foreach \x in {#1}
        \path (\x) node {X};
}
\raggedright \textbf{SOLUTION}:\vspace{2mm}
\raggedright Steps to solve expression using K-map:\\1.Select K-map according to the number of variables.
\\2.Identify minterms or maxterms as given in problem.
\\3.For SOP put $1's$ in blocks of K-map respective to the minterms.
\\4.For POS put $0's$ in blocks of K-map respective to the maxterms.
\\5.Make rectangular groups containing total terms in power of two like 2,4,8 ..(except 1) and try to cover as many elements as you can in one group.
\\6.From the groups made in step 5 find the product terms and sum them up for SOP form.


   \center \begin{Karnaugh}
        \contingut{1,1,1,1,1,1,0,0,0,0,1,1,0,0,1,0}
    
    \end{Karnaugh}

 \center  \begin{Karnaugh}
        \contingut{1,1,1,1,1,1,0,0,0,0,1,1,0,0,1,0}
        
       \implicant{0}{5}{purple}
       
    \end{Karnaugh}
\\ \centering  \textbf{Minterm-1:}\hspace{2mm}$\overline{XZ}$\\
 
 \center   \begin{Karnaugh}
        \contingut{1,1,1,1,1,1,0,0,0,0,1,1,0,0,1,0}
       \implicantdaltbaix[3pt]{3}{10}{blue}
    \end{Karnaugh}
\\ \centering \textbf{Minterm-2:}\hspace{2mm}$\overline{Y}$Z  \\
   
    \center \begin{Karnaugh}
        \contingut{1,1,1,1,1,1,0,0,0,0,1,1,0,0,1,0}
       \implicant{14}{10}{green}
    \end{Karnaugh}  
\\ \centering \textbf{Minterm-3:}\hspace{2mm}XZ$\overline{W}$ \\ 
 \center \begin{Karnaugh}
        \contingut{1,1,1,1,1,1,0,0,0,0,1,1,0,0,1,0}
      % \implicant{1}{14}{black}
       \implicant{0}{5}{purple}
       \implicant{14}{10}{green}
       \implicantdaltbaix[3pt]{3}{10}{blue}
       %\implicantcostats{4}{14}{green}
    \end{Karnaugh}
   
 \textbf{Expression}: $\overline{XZ}$+$\overline{Y}$Z+XZ$\overline{W}$ \vspace{5mm}\\
\raggedright Truth table for the boolean expression $\overline{XZ}$+$\overline{Y}$Z+XZ$\overline{W}$ is given below:

  \begin{center}
    \label{tab:truthtable}
    \setlength{\arrayrulewidth}{0.5mm}
\setlength{\tabcolsep}{18pt}
\renewcommand{\arraystretch}{1.5}
    \begin{tabular}{|l|c|r|l|c|}
    \hline % <-- Alignments: 1st column left, 2nd middle and 3rd right, with vertical lines in between
      \textbf{X} & \textbf{Y} & \textbf{Z} & \textbf{W} & \textbf{F}\\
      \hline
      0 & 0 & 0 & 0 & 1\\
      0 & 0 & 0 & 1 & 1\\
      0 & 0 & 1 & 0 & 1\\
      0 & 0 & 1 & 1 & 1\\
      0 & 1 & 0 & 0 & 1\\
      0 & 1 & 0 & 1 & 1\\
      0 & 1 & 1 & 0 & 0\\
      0 & 1 & 1 & 1 & 0\\
      1 & 0 & 0 & 0 & 0\\
      1 & 0 & 0 & 1 & 0\\
      1 & 0 & 1 & 0 & 1\\
      1 & 0 & 1 & 1 & 1\\
      1 & 1 & 0 & 0 & 0\\
      1 & 1 & 0 & 1 & 0\\
      1 & 1 & 1 & 0 & 1\\
      1 & 1 & 1 & 1 & 0\\
      \hline
      
    \end{tabular}
  \end{center}

 
 \raggedright The above truth table can be verified in arm.\\1.consider any 4 digital pins as inputs and take different combinations of 4 bit input.\\2.Make one of the digitalpins of the arm as output.\\3. Connect LED to the output pin \\4. The connections are given in the table below:

 \begin{center}
 \setlength{\arrayrulewidth}{0.5mm}
\setlength{\tabcolsep}{15pt}
\renewcommand{\arraystretch}{1.5}
    \begin{tabular}{|l|c|c|}
    \hline 
    \textbf{Arm pins} & \textbf{Input} & \textbf{output} \\
    \hline
    2 & +vcc/gnd & -\\
    3 & +vcc/gnd & -\\
    4 & +vcc/gnd & -\\
    5 & +vcc/gnd & -\\
    8 & - & LED\\
    \hline
      \end{tabular}
  \end{center}
\raggedright 5. download the code from the link below and upload into the arm\\
Github link: \href{https://github.com/lakshmikamakshi/FWC/tree/main/assignment_1/codes}{Assignment_arm}.



  \end{multicols}
\end{document}
